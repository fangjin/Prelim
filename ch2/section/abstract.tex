Quantifying information diffusion on social network has been an interesting and unresolved problem for several years now. A better understanding of information diffusion, especially how news and rumors propagate through a network empower us to design strategies that can enhance spreading of news and curbing of rumors. Epidemic models have been used in the past to study information diffusion based on an assumption that rumor/news spreading is no different than the propagation of a contagious disease.

In this paper, we use an enhanced epidemic model SEIZ that has been specifically designed for information diffusion. The model introduces one more compartment called exposed (E), which refers to the individuals who has been exposed to a story but have still not adopted/rejected it. We use five true news stories and three rumors from varied geographical locations and topics. We also introduce a one-step graph transfer model that can mimic step by step information propagation on Twitter. Our experimental results prove that SEIZ model is far more accurate in describing information diffusion than the other baseline epidemic models. Further, our one-step graph transfer model imitates information cascades of the stories with a very reasonable error. 

%Epidemiological models have been successfully used to study disease progression for many years. The SIR (Susceptible, Infected, and Recovery) model has been used timelessly over the past 90 years, focused on disease progression. More recent applications include vaccination delivery strategy. In this paper, we examine the use of traditional epidemiological models applied to twitter data in the hopes of examining topic-based news progression pattern. 
%
%There are many similarities between disease progression and news and rumor propagation. However, many differences remain which requiring different models and implementation strategies to apply these historically disease-based models to information flow applications. We examine these questions with two models, the SIS (Susceptible and Infected) model, and a modern SEIZ (Susceptible, Exposed, Infected, Skeptics) model, by comparing the results we found the SEIZ population model fit better than SIS model for the topic-based news propagation over twitter. In this paper, we try to find some dynamics and evolution patterns for the topic-based news spreading over twitter, and for the future direction, hopefully we can get some trend prediction based on this work.
