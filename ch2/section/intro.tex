\section{Introduction}
Online social networks have become a staging ground for modern
movements, with the Arab Spring being the most prominent example.
Nine out of ten Egyptians and Tunisians responded to a poll indicating
that they used Facebook to organize protests and spread awareness.
%\footnote{http://www.thenational.ae/news/uae-news/facebook-and-twitter-key-to-arab-spring-uprisings-report}.
As a precautionary measure, governments have taken to
blocking social networking websites, showcasing the importance of understanding
this phenomenon.

Interestingly, the role of social networks is not limited to
helping organize the activities of disruptive elements.
Many key government and news
agencies have also begun to embrace Twitter and other social platforms
to disseminate information.
After the tragic 2013 explosions at the Boston Marathon, the
FBI resorted to online social networks to broadcast crucial information
about the suspects. The viral diffusion of information provided them with
vital information about the suspects.
At the same time it is well known
that online activity on sites
such as Reddit led to mistaken identification of some individuals
and the spread of several rumors.

We were motivated to apply the latest in epidemiological modeling to understand
information diffusion on Twitter, in relation to the spread
of both news and rumors.
Epidemiological models provide
a classical approach to study how information diffuses.
These models typically divide the total population into several compartments
which reflect the status of an individual. For instance, common compartments
denote susceptible (S), exposed (E), infected (I), and
recovered (R) individuals. Individuals transit from one compartment to another, with
certain probabilities that have to be estimated from data.
The simplest model, SI, has two states; susceptible (S) individuals get infected (I) by one of their neighbors
and stay infected thereinafter. While conceptually easy to understand,
it is also unrealistic
for practical situations. The SIS model is
popular in infectious disease modeling wherein individuals can transition back
and forth between
susceptible (S) and infected (I) states (e.g., think of allergies and
the common cold); this model is often used as the
baseline model for more sophisticated approaches. The SIR model
enables individuals to
recover (R) but is not suited for modeling news cascades on Twitter since there
is no intuitive mapping to what `recovering' means.
The SEIZ
model (susceptible, exposed, infected, skeptic) proposed by Bettencourt et al.~\cite{powerofgoodidea:2006}
takes the interesting approach of introducing an exposed state (E).
Individuals in such a state take some time before they begin to believe (I)
in a story (i.e., get infected). While the authors
of~\cite{powerofgoodidea:2006}
used this approach to model the adoption of Feynman
diagrams by communities of physicists, our work explores their use in
modeling news and rumors on Twitter.

%{\color{blue} However, the traditional epidemic model SIR (susceptible, infected, Recovery) is not quite suitable for Twitter propagation since once people accept an idea, it is hard to ``recovery". Hence, in this paper for news spreading, we don't consider ``R".}

The key contributions of this paper are:

\begin{itemize}
\item Our work is the first to employ the SEIZ model to model real Twitter
datasets. We employ non-linear least squares optimization of the underlying
systems of ODEs over
tweet data, and demonstrate how this model is better at modeling
rumor and news diffusion
than the traditional SIS model.
\item We analyze
eight representative stories (four true events and
four rumors) across a range of topics (politics, terrorism, entertainment, and crime) and over several
geographic regions (USA, Mexico, Venezuela, Cuba, Vatican). While not
an exhaustive list, this demonstrates the wide applicability of the
proposed model.
\item We demonstrate the capability of the SEIZ model to quantify
compartment transition dynamics. We showcase how such information
could facilitate the development of screening criteria for distinguishing rumors from real news happenings on Twitter.

\end{itemize}



