\section{Introduction}
This chapter was published in IEEE Computer, Volume 47, Issue 12, pages 90-94, Dec 2014~\cite{jin2014modeling}.

Mark Twain is credited with saying that a lie can travel halfway around the world while the truth is putting on its shoes. As the Ebola disease rages on in West Africa, the only other epidemic being talked about is the rapid spread of misinformation on social media about the disease, its origins and impact, and response strategies. We sought to characterize the spread of both news and rumors on Twitter about the deadly disease with a view to understanding the prevalence of misinformation.

For context, although Ebola is not a new disease, the current outbreak happening in West Africa is believed to be more than three times worse than all the previous Ebola outbreaks in history combined. The three countries that have the most widespread transmission, viz. Guinea, Liberia, and Sierra Leone, are also those where public health experts fear massive under-reporting due to a variety of social considerations. Even syndromic surveillance strategies, e.g., social media mining and participatory surveillance, are not effective in these countries due to poor penetration of Internet use, and lack of roads and communication infrastructure where Ebola is most prevalent.

Social media has become one of the primary sources by which people learn about worldwide developments so it is instructive to study the spread of Ebola related information on Twitter. Most of the current chatter on Twitter about Ebola reached its peak during late Sep-mid Oct (2014) during which period there have been Ebola-related developments in the US and Europe. (In contrast, Twitter penetration in the three specific West African countries is low.)

A brief timeline of these developments will help in the discussion that follows. On September 30, 2014, the CDC confirmed the first importation of Ebola into the United States when Thomas Eric Duncan traveled from Liberia to visit family in Dallas. On October 6, in Madrid, Spain, Teresa Romero, a nurse, was reported to be the first person to have contracted the disease outside of West Africa.

On October 8, back in the US, Duncan succumbed to Ebola. A few days later, a healthcare worker at Texas Presbyterian Hospital in Dallas who provided care for Duncan, tested positive for Ebola. On the morning of Oct. 14, a second healthcare worker, who also provided care for Duncan, reported to the hospital with a low-grade fever and was isolated. This healthcare worker also tested positive for Ebola subsequently.

Many states and cities began making contingency plans and issuing travel advisories and guidelines. Lawmakers called for screening passengers and proposed travel bans for Ebola-stricken countries. On October 23, Craig Spencer, a doctor returning from work in Guinea, was rushed to Bellevue Hospital Center with a 100.3 fever and became New York City�s first Ebola patient.



